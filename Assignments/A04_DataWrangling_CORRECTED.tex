\documentclass[]{article}
\usepackage{lmodern}
\usepackage{amssymb,amsmath}
\usepackage{ifxetex,ifluatex}
\usepackage{fixltx2e} % provides \textsubscript
\ifnum 0\ifxetex 1\fi\ifluatex 1\fi=0 % if pdftex
  \usepackage[T1]{fontenc}
  \usepackage[utf8]{inputenc}
\else % if luatex or xelatex
  \ifxetex
    \usepackage{mathspec}
  \else
    \usepackage{fontspec}
  \fi
  \defaultfontfeatures{Ligatures=TeX,Scale=MatchLowercase}
\fi
% use upquote if available, for straight quotes in verbatim environments
\IfFileExists{upquote.sty}{\usepackage{upquote}}{}
% use microtype if available
\IfFileExists{microtype.sty}{%
\usepackage{microtype}
\UseMicrotypeSet[protrusion]{basicmath} % disable protrusion for tt fonts
}{}
\usepackage[margin=2.54cm]{geometry}
\usepackage{hyperref}
\hypersetup{unicode=true,
            pdftitle={Assignment 4: Data Wrangling},
            pdfauthor={Tristen Townsend},
            pdfborder={0 0 0},
            breaklinks=true}
\urlstyle{same}  % don't use monospace font for urls
\usepackage{color}
\usepackage{fancyvrb}
\newcommand{\VerbBar}{|}
\newcommand{\VERB}{\Verb[commandchars=\\\{\}]}
\DefineVerbatimEnvironment{Highlighting}{Verbatim}{commandchars=\\\{\}}
% Add ',fontsize=\small' for more characters per line
\usepackage{framed}
\definecolor{shadecolor}{RGB}{248,248,248}
\newenvironment{Shaded}{\begin{snugshade}}{\end{snugshade}}
\newcommand{\KeywordTok}[1]{\textcolor[rgb]{0.13,0.29,0.53}{\textbf{#1}}}
\newcommand{\DataTypeTok}[1]{\textcolor[rgb]{0.13,0.29,0.53}{#1}}
\newcommand{\DecValTok}[1]{\textcolor[rgb]{0.00,0.00,0.81}{#1}}
\newcommand{\BaseNTok}[1]{\textcolor[rgb]{0.00,0.00,0.81}{#1}}
\newcommand{\FloatTok}[1]{\textcolor[rgb]{0.00,0.00,0.81}{#1}}
\newcommand{\ConstantTok}[1]{\textcolor[rgb]{0.00,0.00,0.00}{#1}}
\newcommand{\CharTok}[1]{\textcolor[rgb]{0.31,0.60,0.02}{#1}}
\newcommand{\SpecialCharTok}[1]{\textcolor[rgb]{0.00,0.00,0.00}{#1}}
\newcommand{\StringTok}[1]{\textcolor[rgb]{0.31,0.60,0.02}{#1}}
\newcommand{\VerbatimStringTok}[1]{\textcolor[rgb]{0.31,0.60,0.02}{#1}}
\newcommand{\SpecialStringTok}[1]{\textcolor[rgb]{0.31,0.60,0.02}{#1}}
\newcommand{\ImportTok}[1]{#1}
\newcommand{\CommentTok}[1]{\textcolor[rgb]{0.56,0.35,0.01}{\textit{#1}}}
\newcommand{\DocumentationTok}[1]{\textcolor[rgb]{0.56,0.35,0.01}{\textbf{\textit{#1}}}}
\newcommand{\AnnotationTok}[1]{\textcolor[rgb]{0.56,0.35,0.01}{\textbf{\textit{#1}}}}
\newcommand{\CommentVarTok}[1]{\textcolor[rgb]{0.56,0.35,0.01}{\textbf{\textit{#1}}}}
\newcommand{\OtherTok}[1]{\textcolor[rgb]{0.56,0.35,0.01}{#1}}
\newcommand{\FunctionTok}[1]{\textcolor[rgb]{0.00,0.00,0.00}{#1}}
\newcommand{\VariableTok}[1]{\textcolor[rgb]{0.00,0.00,0.00}{#1}}
\newcommand{\ControlFlowTok}[1]{\textcolor[rgb]{0.13,0.29,0.53}{\textbf{#1}}}
\newcommand{\OperatorTok}[1]{\textcolor[rgb]{0.81,0.36,0.00}{\textbf{#1}}}
\newcommand{\BuiltInTok}[1]{#1}
\newcommand{\ExtensionTok}[1]{#1}
\newcommand{\PreprocessorTok}[1]{\textcolor[rgb]{0.56,0.35,0.01}{\textit{#1}}}
\newcommand{\AttributeTok}[1]{\textcolor[rgb]{0.77,0.63,0.00}{#1}}
\newcommand{\RegionMarkerTok}[1]{#1}
\newcommand{\InformationTok}[1]{\textcolor[rgb]{0.56,0.35,0.01}{\textbf{\textit{#1}}}}
\newcommand{\WarningTok}[1]{\textcolor[rgb]{0.56,0.35,0.01}{\textbf{\textit{#1}}}}
\newcommand{\AlertTok}[1]{\textcolor[rgb]{0.94,0.16,0.16}{#1}}
\newcommand{\ErrorTok}[1]{\textcolor[rgb]{0.64,0.00,0.00}{\textbf{#1}}}
\newcommand{\NormalTok}[1]{#1}
\usepackage{longtable,booktabs}
\usepackage{graphicx,grffile}
\makeatletter
\def\maxwidth{\ifdim\Gin@nat@width>\linewidth\linewidth\else\Gin@nat@width\fi}
\def\maxheight{\ifdim\Gin@nat@height>\textheight\textheight\else\Gin@nat@height\fi}
\makeatother
% Scale images if necessary, so that they will not overflow the page
% margins by default, and it is still possible to overwrite the defaults
% using explicit options in \includegraphics[width, height, ...]{}
\setkeys{Gin}{width=\maxwidth,height=\maxheight,keepaspectratio}
\IfFileExists{parskip.sty}{%
\usepackage{parskip}
}{% else
\setlength{\parindent}{0pt}
\setlength{\parskip}{6pt plus 2pt minus 1pt}
}
\setlength{\emergencystretch}{3em}  % prevent overfull lines
\providecommand{\tightlist}{%
  \setlength{\itemsep}{0pt}\setlength{\parskip}{0pt}}
\setcounter{secnumdepth}{0}
% Redefines (sub)paragraphs to behave more like sections
\ifx\paragraph\undefined\else
\let\oldparagraph\paragraph
\renewcommand{\paragraph}[1]{\oldparagraph{#1}\mbox{}}
\fi
\ifx\subparagraph\undefined\else
\let\oldsubparagraph\subparagraph
\renewcommand{\subparagraph}[1]{\oldsubparagraph{#1}\mbox{}}
\fi

%%% Use protect on footnotes to avoid problems with footnotes in titles
\let\rmarkdownfootnote\footnote%
\def\footnote{\protect\rmarkdownfootnote}

%%% Change title format to be more compact
\usepackage{titling}

% Create subtitle command for use in maketitle
\newcommand{\subtitle}[1]{
  \posttitle{
    \begin{center}\large#1\end{center}
    }
}

\setlength{\droptitle}{-2em}

  \title{Assignment 4: Data Wrangling}
    \pretitle{\vspace{\droptitle}\centering\huge}
  \posttitle{\par}
    \author{Tristen Townsend}
    \preauthor{\centering\large\emph}
  \postauthor{\par}
    \date{}
    \predate{}\postdate{}
  

\begin{document}
\maketitle

\subsection{OVERVIEW}\label{overview}

This exercise accompanies the lessons in Environmental Data Analytics
(ENV872L) on data wrangling.

\subsection{Directions}\label{directions}

\begin{enumerate}
\def\labelenumi{\arabic{enumi}.}
\tightlist
\item
  Change ``Student Name'' on line 3 (above) with your name.
\item
  Use the lesson as a guide. It contains code that can be modified to
  complete the assignment.
\item
  Work through the steps, \textbf{creating code and output} that fulfill
  each instruction.
\item
  Be sure to \textbf{answer the questions} in this assignment document.
  Space for your answers is provided in this document and is indicated
  by the ``\textgreater{}'' character. If you need a second paragraph be
  sure to start the first line with ``\textgreater{}''. You should
  notice that the answer is highlighted in green by RStudio.
\item
  When you have completed the assignment, \textbf{Knit} the text and
  code into a single PDF file. You will need to have the correct
  software installed to do this (see Software Installation Guide) Press
  the \texttt{Knit} button in the RStudio scripting panel. This will
  save the PDF output in your Assignments folder.
\item
  After Knitting, please submit the completed exercise (PDF file) to the
  dropbox in Sakai. Please add your last name into the file name (e.g.,
  ``Salk\_A04\_DataWrangling.pdf'') prior to submission.
\end{enumerate}

The completed exercise is due on Thursday, 7 February, 2019 before class
begins.

\subsection{Set up your session}\label{set-up-your-session}

\begin{enumerate}
\def\labelenumi{\arabic{enumi}.}
\item
  Check your working directory, load the \texttt{tidyverse} package, and
  upload all four raw data files associated with the EPA Air dataset.
  See the README file for the EPA air datasets for more information
  (especially if you have not worked with air quality data previously).
\item
  Generate a few lines of code to get to know your datasets (basic data
  summaries, etc.).
\end{enumerate}

\begin{Shaded}
\begin{Highlighting}[]
\CommentTok{#1}
\KeywordTok{getwd}\NormalTok{()}
\KeywordTok{library}\NormalTok{(tidyverse)}
\end{Highlighting}
\end{Shaded}

\begin{verbatim}
## -- Attaching packages ------------------- tidyverse 1.2.1 --
\end{verbatim}

\begin{verbatim}
## v ggplot2 3.1.0     v purrr   0.2.5
## v tibble  2.0.1     v dplyr   0.7.8
## v tidyr   0.8.2     v stringr 1.3.1
## v readr   1.3.1     v forcats 0.3.0
\end{verbatim}

\begin{verbatim}
## Warning: package 'tibble' was built under R version 3.5.2
\end{verbatim}

\begin{verbatim}
## -- Conflicts ---------------------- tidyverse_conflicts() --
## x dplyr::filter() masks stats::filter()
## x dplyr::lag()    masks stats::lag()
\end{verbatim}

\begin{Shaded}
\begin{Highlighting}[]
\KeywordTok{library}\NormalTok{(dplyr)}
\KeywordTok{library}\NormalTok{(forcats)}
\KeywordTok{library}\NormalTok{(lubridate)}
\end{Highlighting}
\end{Shaded}

\begin{verbatim}
## 
## Attaching package: 'lubridate'
\end{verbatim}

\begin{verbatim}
## The following object is masked from 'package:base':
## 
##     date
\end{verbatim}

\begin{Shaded}
\begin{Highlighting}[]
\KeywordTok{library}\NormalTok{(pander)}

\NormalTok{EPA.ozone.}\DecValTok{17}\NormalTok{ <-}\StringTok{ }\KeywordTok{read.csv}\NormalTok{(}\StringTok{"./Data/Raw/EPAair_O3_NC2017_raw.csv"}\NormalTok{)}
\NormalTok{EPA.ozone.}\DecValTok{18}\NormalTok{ <-}\StringTok{ }\KeywordTok{read.csv}\NormalTok{(}\StringTok{"./Data/Raw/EPAair_O3_NC2018_raw.csv"}\NormalTok{)}
\NormalTok{EPA.pm25.}\DecValTok{17}\NormalTok{ <-}\StringTok{ }\KeywordTok{read.csv}\NormalTok{(}\StringTok{"./Data/Raw/EPAair_PM25_NC2017_raw.csv"}\NormalTok{)}
\NormalTok{EPA.pm25.}\DecValTok{18}\NormalTok{ <-}\StringTok{ }\KeywordTok{read.csv}\NormalTok{(}\StringTok{"./Data/Raw/EPAair_PM25_NC2018_raw.csv"}\NormalTok{)}

\CommentTok{#2}
\KeywordTok{head}\NormalTok{(EPA.ozone.}\DecValTok{17}\NormalTok{)}
\KeywordTok{colnames}\NormalTok{(EPA.ozone.}\DecValTok{17}\NormalTok{)}
\KeywordTok{summary}\NormalTok{(EPA.ozone.}\DecValTok{17}\NormalTok{)}
\KeywordTok{dim}\NormalTok{(EPA.ozone.}\DecValTok{17}\NormalTok{)}

\KeywordTok{head}\NormalTok{(EPA.ozone.}\DecValTok{18}\NormalTok{)}
\KeywordTok{colnames}\NormalTok{(EPA.ozone.}\DecValTok{18}\NormalTok{)}
\KeywordTok{summary}\NormalTok{(EPA.ozone.}\DecValTok{18}\NormalTok{)}
\KeywordTok{dim}\NormalTok{(EPA.ozone.}\DecValTok{18}\NormalTok{)}

\KeywordTok{head}\NormalTok{(EPA.pm25.}\DecValTok{17}\NormalTok{)}
\KeywordTok{colnames}\NormalTok{(EPA.pm25.}\DecValTok{17}\NormalTok{)}
\KeywordTok{summary}\NormalTok{(EPA.pm25.}\DecValTok{17}\NormalTok{)}
\KeywordTok{dim}\NormalTok{(EPA.pm25.}\DecValTok{17}\NormalTok{)}

\KeywordTok{head}\NormalTok{(EPA.pm25.}\DecValTok{18}\NormalTok{)}
\KeywordTok{colnames}\NormalTok{(EPA.pm25.}\DecValTok{18}\NormalTok{)}
\KeywordTok{summary}\NormalTok{(EPA.pm25.}\DecValTok{18}\NormalTok{)}
\KeywordTok{dim}\NormalTok{(EPA.pm25.}\DecValTok{18}\NormalTok{)}
\end{Highlighting}
\end{Shaded}

\subsection{Wrangle individual datasets to create processed
files.}\label{wrangle-individual-datasets-to-create-processed-files.}

\begin{enumerate}
\def\labelenumi{\arabic{enumi}.}
\setcounter{enumi}{2}
\tightlist
\item
  Change date to date
\item
  Select the following columns: Date, DAILY\_AQI\_VALUE, Site.Name,
  AQS\_PARAMETER\_DESC, COUNTY, SITE\_LATITUDE, SITE\_LONGITUDE
\item
  For the PM2.5 datasets, fill all cells in AQS\_PARAMETER\_DESC with
  ``PM2.5'' (all cells in this column should be identical).
\item
  Save all four processed datasets in the Processed folder.
\end{enumerate}

\begin{Shaded}
\begin{Highlighting}[]
\CommentTok{#3}
\KeywordTok{class}\NormalTok{(EPA.ozone.}\DecValTok{17}\OperatorTok{$}\NormalTok{Date)}
\end{Highlighting}
\end{Shaded}

\begin{verbatim}
## [1] "factor"
\end{verbatim}

\begin{Shaded}
\begin{Highlighting}[]
\KeywordTok{class}\NormalTok{(EPA.ozone.}\DecValTok{18}\OperatorTok{$}\NormalTok{Date)}
\end{Highlighting}
\end{Shaded}

\begin{verbatim}
## [1] "factor"
\end{verbatim}

\begin{Shaded}
\begin{Highlighting}[]
\KeywordTok{class}\NormalTok{(EPA.pm25.}\DecValTok{17}\OperatorTok{$}\NormalTok{Date)}
\end{Highlighting}
\end{Shaded}

\begin{verbatim}
## [1] "factor"
\end{verbatim}

\begin{Shaded}
\begin{Highlighting}[]
\KeywordTok{class}\NormalTok{(EPA.pm25.}\DecValTok{18}\OperatorTok{$}\NormalTok{Date)}
\end{Highlighting}
\end{Shaded}

\begin{verbatim}
## [1] "factor"
\end{verbatim}

\begin{Shaded}
\begin{Highlighting}[]
\NormalTok{EPA.ozone.}\DecValTok{17}\OperatorTok{$}\NormalTok{Date <-}\StringTok{ }\KeywordTok{as.Date}\NormalTok{(EPA.ozone.}\DecValTok{17}\OperatorTok{$}\NormalTok{Date, }\DataTypeTok{format =} \StringTok{"%m/%d/%y"}\NormalTok{)}
\NormalTok{EPA.ozone.}\DecValTok{18}\OperatorTok{$}\NormalTok{Date <-}\StringTok{ }\KeywordTok{as.Date}\NormalTok{(EPA.ozone.}\DecValTok{18}\OperatorTok{$}\NormalTok{Date, }\DataTypeTok{format =} \StringTok{"%m/%d/%y"}\NormalTok{)}
\NormalTok{EPA.pm25.}\DecValTok{17}\OperatorTok{$}\NormalTok{Date <-}\StringTok{ }\KeywordTok{as.Date}\NormalTok{(EPA.pm25.}\DecValTok{17}\OperatorTok{$}\NormalTok{Date, }\DataTypeTok{format =} \StringTok{"%m/%d/%y"}\NormalTok{)}
\NormalTok{EPA.pm25.}\DecValTok{18}\OperatorTok{$}\NormalTok{Date <-}\StringTok{ }\KeywordTok{as.Date}\NormalTok{(EPA.pm25.}\DecValTok{18}\OperatorTok{$}\NormalTok{Date, }\DataTypeTok{format =} \StringTok{"%m/%d/%y"}\NormalTok{)}

\CommentTok{#4}
\NormalTok{EPA.ozone.}\FloatTok{17.}\NormalTok{processed <-}\StringTok{ }\KeywordTok{select}\NormalTok{(EPA.ozone.}\DecValTok{17}\NormalTok{, }\StringTok{"Date"}\NormalTok{, }\StringTok{"DAILY_AQI_VALUE"}\NormalTok{, }
  \StringTok{"Site.Name"}\NormalTok{, }\StringTok{"AQS_PARAMETER_DESC"}\NormalTok{, }\StringTok{"COUNTY"}\NormalTok{, }\StringTok{"SITE_LATITUDE"}\NormalTok{, }\StringTok{"SITE_LONGITUDE"}\NormalTok{)}

\NormalTok{EPA.ozone.}\FloatTok{18.}\NormalTok{processed <-}\StringTok{ }\KeywordTok{select}\NormalTok{(EPA.ozone.}\DecValTok{18}\NormalTok{, }\StringTok{"Date"}\NormalTok{, }\StringTok{"DAILY_AQI_VALUE"}\NormalTok{, }
  \StringTok{"Site.Name"}\NormalTok{, }\StringTok{"AQS_PARAMETER_DESC"}\NormalTok{, }\StringTok{"COUNTY"}\NormalTok{, }\StringTok{"SITE_LATITUDE"}\NormalTok{, }\StringTok{"SITE_LONGITUDE"}\NormalTok{)}

\NormalTok{EPA.pm25.}\FloatTok{17.}\NormalTok{processed <-}\StringTok{ }\KeywordTok{select}\NormalTok{(EPA.pm25.}\DecValTok{17}\NormalTok{, }\StringTok{"Date"}\NormalTok{, }\StringTok{"DAILY_AQI_VALUE"}\NormalTok{, }
  \StringTok{"Site.Name"}\NormalTok{, }\StringTok{"AQS_PARAMETER_DESC"}\NormalTok{, }\StringTok{"COUNTY"}\NormalTok{, }\StringTok{"SITE_LATITUDE"}\NormalTok{, }\StringTok{"SITE_LONGITUDE"}\NormalTok{)}

\NormalTok{EPA.pm25.}\FloatTok{18.}\NormalTok{processed <-}\StringTok{ }\KeywordTok{select}\NormalTok{(EPA.pm25.}\DecValTok{18}\NormalTok{, }\StringTok{"Date"}\NormalTok{, }\StringTok{"DAILY_AQI_VALUE"}\NormalTok{, }
  \StringTok{"Site.Name"}\NormalTok{, }\StringTok{"AQS_PARAMETER_DESC"}\NormalTok{, }\StringTok{"COUNTY"}\NormalTok{, }\StringTok{"SITE_LATITUDE"}\NormalTok{, }\StringTok{"SITE_LONGITUDE"}\NormalTok{)}

\CommentTok{#5}
\NormalTok{EPA.pm25.}\FloatTok{17.}\NormalTok{processed}\OperatorTok{$}\NormalTok{AQS_PARAMETER_DESC <-}\StringTok{ "PM2.5"}
\NormalTok{EPA.pm25.}\FloatTok{18.}\NormalTok{processed}\OperatorTok{$}\NormalTok{AQS_PARAMETER_DESC <-}\StringTok{ "PM2.5"}

\CommentTok{#6}
\KeywordTok{write.csv}\NormalTok{(EPA.ozone.}\FloatTok{17.}\NormalTok{processed, }\DataTypeTok{row.names =} \OtherTok{FALSE}\NormalTok{, }
          \DataTypeTok{file =}\StringTok{"./Data/Processed/EPAair_O3_NC2017_processed.csv"}\NormalTok{)}

\KeywordTok{write.csv}\NormalTok{(EPA.ozone.}\FloatTok{18.}\NormalTok{processed, }\DataTypeTok{row.names =} \OtherTok{FALSE}\NormalTok{, }
          \DataTypeTok{file =}\StringTok{"./Data/Processed/EPAair_O3_NC2018_processed.csv"}\NormalTok{)}

\KeywordTok{write.csv}\NormalTok{(EPA.pm25.}\FloatTok{17.}\NormalTok{processed, }\DataTypeTok{row.names =} \OtherTok{FALSE}\NormalTok{, }
          \DataTypeTok{file =}\StringTok{"./Data/Processed/EPAair_PM25_NC2017_processed.csv"}\NormalTok{)}

\KeywordTok{write.csv}\NormalTok{(EPA.pm25.}\FloatTok{18.}\NormalTok{processed, }\DataTypeTok{row.names =} \OtherTok{FALSE}\NormalTok{, }
          \DataTypeTok{file =}\StringTok{"./Data/Processed/EPAair_PM25_NC2018_processed.csv"}\NormalTok{)}
\end{Highlighting}
\end{Shaded}

\subsection{Combine datasets}\label{combine-datasets}

\begin{enumerate}
\def\labelenumi{\arabic{enumi}.}
\setcounter{enumi}{6}
\tightlist
\item
  Combine the four datasets with \texttt{rbind}. Make sure your column
  names are identical prior to running this code.
\item
  Wrangle your new dataset with a pipe function (\%\textgreater{}\%) so
  that it fills the following conditions:
\end{enumerate}

\begin{itemize}
\tightlist
\item
  Sites: Blackstone, Bryson City, Triple Oak
\item
  Add columns for ``Month'' and ``Year'' by parsing your ``Date'' column
  (hint: \texttt{separate} function or \texttt{lubridate} package)
\end{itemize}

\begin{enumerate}
\def\labelenumi{\arabic{enumi}.}
\setcounter{enumi}{8}
\tightlist
\item
  Spread your datasets such that AQI values for ozone and PM2.5 are in
  separate columns. Each location on a specific date should now occupy
  only one row.
\item
  Call up the dimensions of your new tidy dataset.
\item
  Save your processed dataset with the following file name:
  ``EPAair\_O3\_PM25\_NC1718\_Processed.csv''
\end{enumerate}

\begin{Shaded}
\begin{Highlighting}[]
\CommentTok{#7}
\NormalTok{EPA.ozone.pm25.}\FloatTok{1718.}\NormalTok{combined <-}\StringTok{ }
\StringTok{  }\KeywordTok{rbind}\NormalTok{(EPA.ozone.}\FloatTok{17.}\NormalTok{processed, EPA.ozone.}\FloatTok{18.}\NormalTok{processed, }
\NormalTok{        EPA.pm25.}\FloatTok{17.}\NormalTok{processed, EPA.pm25.}\FloatTok{18.}\NormalTok{processed)}

\CommentTok{#8}
\NormalTok{EPA.ozone.pm25.}\FloatTok{1718.}\NormalTok{combined <-}
\StringTok{  }\NormalTok{EPA.ozone.pm25.}\FloatTok{1718.}\NormalTok{combined }\OperatorTok
\StringTok{  }\KeywordTok{filter}\NormalTok{(Site.Name }\OperatorTok{==}\StringTok{ "Blackstone"}\OperatorTok{|}\StringTok{ }\NormalTok{Site.Name }\OperatorTok{==}\StringTok{ "Bryson City"} \OperatorTok{|}\StringTok{ }
\StringTok{          }\NormalTok{Site.Name }\OperatorTok{==}\StringTok{ "Triple Oak"}\NormalTok{) }\OperatorTok\StringTok{  }
\StringTok{  }\KeywordTok{mutate_at}\NormalTok{(}\KeywordTok{vars}\NormalTok{(Date), }\KeywordTok{funs}\NormalTok{(year, month))}

\CommentTok{#9}
\NormalTok{EPA.ozone.pm25.}\FloatTok{1718.}\NormalTok{spread <-}\StringTok{ }\KeywordTok{spread}\NormalTok{(EPA.ozone.pm25.}\FloatTok{1718.}\NormalTok{combined, }
\NormalTok{  AQS_PARAMETER_DESC, DAILY_AQI_VALUE)}

\CommentTok{#10}
\KeywordTok{dim}\NormalTok{(EPA.ozone.pm25.}\FloatTok{1718.}\NormalTok{spread)}
\end{Highlighting}
\end{Shaded}

\begin{verbatim}
## [1] 1953    9
\end{verbatim}

\begin{Shaded}
\begin{Highlighting}[]
\CommentTok{#11}
\KeywordTok{write.csv}\NormalTok{(EPA.ozone.pm25.}\FloatTok{1718.}\NormalTok{spread, }\DataTypeTok{row.names =} \OtherTok{FALSE}\NormalTok{, }
          \DataTypeTok{file =}\StringTok{"./Data/Processed/EPAair_O3_PM25_NC1718_Processed.csv"}\NormalTok{)}
\end{Highlighting}
\end{Shaded}

\subsection{Generate summary tables}\label{generate-summary-tables}

\begin{enumerate}
\def\labelenumi{\arabic{enumi}.}
\setcounter{enumi}{11}
\tightlist
\item
  Use the split-apply-combine strategy to generate two new data frames:
\end{enumerate}

\begin{enumerate}
\def\labelenumi{\alph{enumi}.}
\tightlist
\item
  A summary table of mean AQI values for O3 and PM2.5 by month
\item
  A summary table of the mean, minimum, and maximum AQI values of O3 and
  PM2.5 for each site
\end{enumerate}

\begin{enumerate}
\def\labelenumi{\arabic{enumi}.}
\setcounter{enumi}{12}
\tightlist
\item
  Display the data frames.
\end{enumerate}

\begin{Shaded}
\begin{Highlighting}[]
\CommentTok{#12a}
\NormalTok{EPA.ozone.pm25.}\FloatTok{1718.}\NormalTok{summaries.month <-}\StringTok{ }
\StringTok{  }\NormalTok{EPA.ozone.pm25.}\FloatTok{1718.}\NormalTok{spread  }\OperatorTok\StringTok{ }
\StringTok{  }\KeywordTok{group_by}\NormalTok{(month) }\OperatorTok\StringTok{ }
\StringTok{  }\KeywordTok{filter}\NormalTok{(}\OperatorTok{!}\KeywordTok{is.na}\NormalTok{(Ozone) }\OperatorTok{&}\StringTok{ }\OperatorTok{!}\KeywordTok{is.na}\NormalTok{(PM2.}\DecValTok{5}\NormalTok{)) }\OperatorTok
\StringTok{  }\KeywordTok{summarise}\NormalTok{(}
    \DataTypeTok{Monthly.Ozone.Averages =} \KeywordTok{mean}\NormalTok{(Ozone),}
    \DataTypeTok{Monthly.PM25.Averages =} \KeywordTok{mean}\NormalTok{(PM2.}\DecValTok{5}\NormalTok{)}
\NormalTok{    )}

\CommentTok{#12b}
\NormalTok{EPA.ozone.pm25.}\FloatTok{1718.}\NormalTok{summaries.site <-}\StringTok{ }
\StringTok{  }\NormalTok{EPA.ozone.pm25.}\FloatTok{1718.}\NormalTok{spread  }\OperatorTok\StringTok{ }
\StringTok{  }\KeywordTok{group_by}\NormalTok{(Site.Name) }\OperatorTok\StringTok{ }
\StringTok{  }\KeywordTok{filter}\NormalTok{(}\OperatorTok{!}\KeywordTok{is.na}\NormalTok{(Ozone) }\OperatorTok{&}\StringTok{ }\OperatorTok{!}\KeywordTok{is.na}\NormalTok{(PM2.}\DecValTok{5}\NormalTok{)) }\OperatorTok
\StringTok{  }\KeywordTok{summarise}\NormalTok{(}
    \DataTypeTok{Ozone.Average.By.Site =} \KeywordTok{mean}\NormalTok{(Ozone),}
    \DataTypeTok{Ozone.Min.By.Site =} \KeywordTok{min}\NormalTok{(Ozone),}
    \DataTypeTok{Ozone.Max.By.Site =} \KeywordTok{max}\NormalTok{(Ozone),}
    \DataTypeTok{PM25.Average.By.Site=} \KeywordTok{mean}\NormalTok{(PM2.}\DecValTok{5}\NormalTok{),}
    \DataTypeTok{PM25.Min.By.Site =} \KeywordTok{min}\NormalTok{(PM2.}\DecValTok{5}\NormalTok{),}
    \DataTypeTok{PM25.Max.By.Site =} \KeywordTok{max}\NormalTok{(PM2.}\DecValTok{5}\NormalTok{)}
\NormalTok{    )}


\CommentTok{#13}
\KeywordTok{pander}\NormalTok{(EPA.ozone.pm25.}\FloatTok{1718.}\NormalTok{summaries.month, }\DataTypeTok{type =} \StringTok{'grid'}\NormalTok{)}
\end{Highlighting}
\end{Shaded}

\begin{longtable}[]{@{}ccc@{}}
\toprule
\begin{minipage}[b]{0.10\columnwidth}\centering\strut
month\strut
\end{minipage} & \begin{minipage}[b]{0.32\columnwidth}\centering\strut
Monthly.Ozone.Averages\strut
\end{minipage} & \begin{minipage}[b]{0.32\columnwidth}\centering\strut
Monthly.PM25.Averages\strut
\end{minipage}\tabularnewline
\midrule
\endhead
\begin{minipage}[t]{0.10\columnwidth}\centering\strut
1\strut
\end{minipage} & \begin{minipage}[t]{0.32\columnwidth}\centering\strut
31.48\strut
\end{minipage} & \begin{minipage}[t]{0.32\columnwidth}\centering\strut
34.24\strut
\end{minipage}\tabularnewline
\begin{minipage}[t]{0.10\columnwidth}\centering\strut
2\strut
\end{minipage} & \begin{minipage}[t]{0.32\columnwidth}\centering\strut
35.41\strut
\end{minipage} & \begin{minipage}[t]{0.32\columnwidth}\centering\strut
37.57\strut
\end{minipage}\tabularnewline
\begin{minipage}[t]{0.10\columnwidth}\centering\strut
3\strut
\end{minipage} & \begin{minipage}[t]{0.32\columnwidth}\centering\strut
42.4\strut
\end{minipage} & \begin{minipage}[t]{0.32\columnwidth}\centering\strut
37.41\strut
\end{minipage}\tabularnewline
\begin{minipage}[t]{0.10\columnwidth}\centering\strut
4\strut
\end{minipage} & \begin{minipage}[t]{0.32\columnwidth}\centering\strut
43.49\strut
\end{minipage} & \begin{minipage}[t]{0.32\columnwidth}\centering\strut
31.52\strut
\end{minipage}\tabularnewline
\begin{minipage}[t]{0.10\columnwidth}\centering\strut
5\strut
\end{minipage} & \begin{minipage}[t]{0.32\columnwidth}\centering\strut
39.49\strut
\end{minipage} & \begin{minipage}[t]{0.32\columnwidth}\centering\strut
30.63\strut
\end{minipage}\tabularnewline
\begin{minipage}[t]{0.10\columnwidth}\centering\strut
6\strut
\end{minipage} & \begin{minipage}[t]{0.32\columnwidth}\centering\strut
39.17\strut
\end{minipage} & \begin{minipage}[t]{0.32\columnwidth}\centering\strut
30.92\strut
\end{minipage}\tabularnewline
\begin{minipage}[t]{0.10\columnwidth}\centering\strut
7\strut
\end{minipage} & \begin{minipage}[t]{0.32\columnwidth}\centering\strut
38.33\strut
\end{minipage} & \begin{minipage}[t]{0.32\columnwidth}\centering\strut
31.93\strut
\end{minipage}\tabularnewline
\begin{minipage}[t]{0.10\columnwidth}\centering\strut
8\strut
\end{minipage} & \begin{minipage}[t]{0.32\columnwidth}\centering\strut
34.4\strut
\end{minipage} & \begin{minipage}[t]{0.32\columnwidth}\centering\strut
32.34\strut
\end{minipage}\tabularnewline
\begin{minipage}[t]{0.10\columnwidth}\centering\strut
9\strut
\end{minipage} & \begin{minipage}[t]{0.32\columnwidth}\centering\strut
32.64\strut
\end{minipage} & \begin{minipage}[t]{0.32\columnwidth}\centering\strut
30.65\strut
\end{minipage}\tabularnewline
\begin{minipage}[t]{0.10\columnwidth}\centering\strut
10\strut
\end{minipage} & \begin{minipage}[t]{0.32\columnwidth}\centering\strut
32.29\strut
\end{minipage} & \begin{minipage}[t]{0.32\columnwidth}\centering\strut
30.13\strut
\end{minipage}\tabularnewline
\begin{minipage}[t]{0.10\columnwidth}\centering\strut
11\strut
\end{minipage} & \begin{minipage}[t]{0.32\columnwidth}\centering\strut
30.07\strut
\end{minipage} & \begin{minipage}[t]{0.32\columnwidth}\centering\strut
42.14\strut
\end{minipage}\tabularnewline
\begin{minipage}[t]{0.10\columnwidth}\centering\strut
12\strut
\end{minipage} & \begin{minipage}[t]{0.32\columnwidth}\centering\strut
29.78\strut
\end{minipage} & \begin{minipage}[t]{0.32\columnwidth}\centering\strut
46.62\strut
\end{minipage}\tabularnewline
\bottomrule
\end{longtable}

\begin{Shaded}
\begin{Highlighting}[]
\KeywordTok{pander}\NormalTok{(EPA.ozone.pm25.}\FloatTok{1718.}\NormalTok{summaries.site, }\DataTypeTok{type =} \StringTok{'grid'}\NormalTok{)}
\end{Highlighting}
\end{Shaded}

\begin{longtable}[]{@{}cccc@{}}
\caption{Table continues below}\tabularnewline
\toprule
\begin{minipage}[b]{0.16\columnwidth}\centering\strut
Site.Name\strut
\end{minipage} & \begin{minipage}[b]{0.27\columnwidth}\centering\strut
Ozone.Average.By.Site\strut
\end{minipage} & \begin{minipage}[b]{0.23\columnwidth}\centering\strut
Ozone.Min.By.Site\strut
\end{minipage} & \begin{minipage}[b]{0.23\columnwidth}\centering\strut
Ozone.Max.By.Site\strut
\end{minipage}\tabularnewline
\midrule
\endfirsthead
\toprule
\begin{minipage}[b]{0.16\columnwidth}\centering\strut
Site.Name\strut
\end{minipage} & \begin{minipage}[b]{0.27\columnwidth}\centering\strut
Ozone.Average.By.Site\strut
\end{minipage} & \begin{minipage}[b]{0.23\columnwidth}\centering\strut
Ozone.Min.By.Site\strut
\end{minipage} & \begin{minipage}[b]{0.23\columnwidth}\centering\strut
Ozone.Max.By.Site\strut
\end{minipage}\tabularnewline
\midrule
\endhead
\begin{minipage}[t]{0.16\columnwidth}\centering\strut
Blackstone\strut
\end{minipage} & \begin{minipage}[t]{0.27\columnwidth}\centering\strut
38.3\strut
\end{minipage} & \begin{minipage}[t]{0.23\columnwidth}\centering\strut
8\strut
\end{minipage} & \begin{minipage}[t]{0.23\columnwidth}\centering\strut
97\strut
\end{minipage}\tabularnewline
\begin{minipage}[t]{0.16\columnwidth}\centering\strut
Bryson City\strut
\end{minipage} & \begin{minipage}[t]{0.27\columnwidth}\centering\strut
35.43\strut
\end{minipage} & \begin{minipage}[t]{0.23\columnwidth}\centering\strut
5\strut
\end{minipage} & \begin{minipage}[t]{0.23\columnwidth}\centering\strut
71\strut
\end{minipage}\tabularnewline
\bottomrule
\end{longtable}

\begin{longtable}[]{@{}ccc@{}}
\toprule
\begin{minipage}[b]{0.29\columnwidth}\centering\strut
PM25.Average.By.Site\strut
\end{minipage} & \begin{minipage}[b]{0.24\columnwidth}\centering\strut
PM25.Min.By.Site\strut
\end{minipage} & \begin{minipage}[b]{0.24\columnwidth}\centering\strut
PM25.Max.By.Site\strut
\end{minipage}\tabularnewline
\midrule
\endhead
\begin{minipage}[t]{0.29\columnwidth}\centering\strut
36.66\strut
\end{minipage} & \begin{minipage}[t]{0.24\columnwidth}\centering\strut
0\strut
\end{minipage} & \begin{minipage}[t]{0.24\columnwidth}\centering\strut
83\strut
\end{minipage}\tabularnewline
\begin{minipage}[t]{0.29\columnwidth}\centering\strut
30.32\strut
\end{minipage} & \begin{minipage}[t]{0.24\columnwidth}\centering\strut
3\strut
\end{minipage} & \begin{minipage}[t]{0.24\columnwidth}\centering\strut
68\strut
\end{minipage}\tabularnewline
\bottomrule
\end{longtable}


\end{document}
